\subsection{Rationale}

Whole genome sequencing projects such as HapMap\cite{HapMap2010}, 1000 Genomes\cite{1000G2012}, UK10K\cite{UK10K} and Genomes of the Netherlands\cite{GoNL} have allowed the generation of large and diverse haplotype reference panels for humans. In comparison the sequencing of individuals within Africa is limited, but there has been a series of whole genome sequencing projects in Africa and a series of data resources have been made available. Our motivation for the African Diversity Reference Panel (ADRP) is to combine all of those resources and create a continent specific reference panel, which will improve imputation accuracy for all studies in Africa based on SNP array technologies or low coverage whole genome sequencing. Our goal is to combine new and existing data sets and generate a large set of haplotypes, which can be easily accessed by everyone for imputation and phasing purposes. The combination of multiple cohorts increases the number of haplotypes and the number of variants, which will both increase the imputation accuracy and the statistical power for any SNP array based GWAS carried out in Africa.
Studies of population genetics and ancient DNA in Africa will also benefit greatly from the ADRP, because Afro-Asiatic, Khoesan, Nilo-Saharan and Southern Bantu ethnolinguistic groups are absent from existing reference panels. The reference panel will therefore provide a resource to better understand the population structure and admixture on the continent.
With the continued decline in genotyping costs and the continuous efforts to design continent specific SNP arrays, which better capture the LD structure in Africa, the reference panel could prove to be a valuable resource to genetic research on the continent.

%With next generation genotyping, and the decline in genotyping costs, it has now become possible to carry out large-scale dense genotyping across individuals for genome wide association studies. While these chip designs capture variation in European populations very well, their utility in capturing variation in more diverse African populations is unclear. Using the African Genome Variation Project resource (AGVP), we have shown that existing chip designs with imputation using large-scale sequencing panels can capture common variation well relative to low coverage sequencing designs. We also show marked improvement in imputation accuracy, even for common variation, with addition of diverse African populations in imputation panels.\cite{Gurdasani2015}

%We hypothesise that development of more efficient chip designs specific to African populations would complement the development of novel reference panels, and improve efficiency of variant capture by genotyping arrays even further. While existing chip designs have included African populations in the ascertainment set, European populations have dominated development of these arrays, and African populations in current reference panels are not representative of more differentiated population groups within Africa. Here, we describe the development of such an array, ascertained on large samples from distinct representative population groups across Africa.