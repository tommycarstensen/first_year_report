\subsection{Summary of Datasets}
The project will incorporate datasets from several population groups collated from different parts of Africa. These include samples collated as part of the AGVP\cite{Gurdasani2015}, (\href{http://www.1000genomes.org}{the 1000 Genomes project})\cite{1000G2012},
% the Genome Diversity in Africa Project (GDAP),
% the Simon’s Foundation (\href{http://www.simonsfoundation.org/life-sciences/simons-genome-diversity-project-dataset/}{http://www.simonsfoundation.org/life-sciences/simons-genome-diversity-project-dataset/}),
 and other shared or publicly available data (figure \ref{fig:map} and table \ref{tab:samples}). These samples have been chosen to be representative of diverse ethnolinguistic and geographical groups across Africa. The imputation accuracy of the reference panel will be evaluated using other populations; e.g. Zulu samples.
 
 As part of the data curation process available SNP array data (table \ref{tab:samples_chip}) will be used for validation of sequenced samples and for generation of a haplotype scaffold for phasing as described on page \pageref{sec:refine_and_phase}.
 
\begin{figure}[htbp]
\centering
 \begin{subfigure}[b]{0.9\textwidth}
  \centering
%  \includegraphics[width=1.0\linewidth]{ADRP_Africa_2015-03-23.jpg}
  \includegraphics[width=0.9\linewidth]{ADRP/figures/Africa} \caption{Population groups on the African continent comprising the reference panel.}
 \end{subfigure}
 \begin{subfigure}[b]{0.9\textwidth}
  \centering
  \includegraphics[width=0.9\linewidth]{ADRP/figures/Uganda.jpg}
  \caption{Southern Uganda south of the tripoint between Northern, Western and Eastern Africa.}
 \end{subfigure}
\caption{Maps showing sample location, sample count and sequencing depth of data to be included.}
\label{fig:map}
\end{figure}

% I need to figure out how to avoid this table from exceeding the page width.
%\begin{landscape}
%%%\begin{longtable}{lllllll}
\begin{table}[htp]
\centering
%\resizebox{\textwidth}{!}{%
\begin{tabular}{llrrllr}
\hline
Country & Ethnolinguistic group & Count & Depth & Source & Location & Size (TB) \\
\hline
%%%\endhead % all the lines above this will be repeated on every page
Uganda & Baganda & 1567 & 4x & UG2G & WTSI & 40.4 \\
Uganda & Banyarwanda & 199 & 4x & UG2G & WTSI & 5.1 \\
Uganda & Rwandese Ugandan & 76 & 4x & UG2G & WTSI & 1.9 \\
Uganda & Barundi & 51 & 4x & UG2G & WTSI & 1.4 \\
Uganda & Banyankole & 36 & 4x & UG2G & WTSI & 0.9 \\
Uganda & Bakiga & 30 & 4x & UG2G & WTSI & 0.8 \\
Uganda & Other & 41 & 4x & UG2G & WTSI & 1.1 \\
Uganda & Baganda & 100 & 4x & AGVP & WTSI & 2.7 \\
South Africa & Zulu & 100 & 4x & AGVP & WTSI & 2.3 \\
Ethiopia & Amhara & 24 & 8x & AGVP & WTSI & 1.0 \\
Ethiopia & Gumuz & 24 & 8x & AGVP & WTSI & 1.0 \\
Ethiopia & Oromo & 24 & 8x & AGVP & WTSI & 1.0 \\
Ethiopia & Somali & 24 & 8x & AGVP & WTSI & 1.0 \\
Ethiopia & Wolayta & 24 & 8x & AGVP & WTSI & 1.0 \\
Egypt & Unspecified & 100 & 8x & GDAP & WTSI & 5.0 \\
South Africa & Khoe-San (Nama) & 104 & 4x & GDAP & WTSI & 3.6 \\
%Uganda & Baganda & 3 & 30x & GDAP & WTSI \\
%South Africa & Zulu & 2 & 30x & GDAP & WTSI \\
%South Africa & Khoe-San (Nama) & 3 & 30x & GDAP & WTSI \\
%Chad & Northern and Southern & 100 & 30x & GDAP & WTSI \\
%Kenya & Kalenjin & 100 & 30x & GDAP & WTSI \\
%Nigeria & Igbo & 100 & 30x & GDAP & WTSI \\
%Ghana & Ashanti & 100 & 30x & GDAP & WTSI \\
%Morocco & Moroccans & 100 & 30x & GDAP & WTSI \\
%Burkina Faso & Gouin, Kaboro, Turka & 100 & 30x & GDAP & WTSI \\
%Gambia & Fula & 100 & 8x & MalariaGEN & ENA \\
Nigeria & Esan (ESN) & 99 & 4x & 1000G & 1000G & 2.2 \\
Gambia & Gambian (GWD) & 113 & 4x & 1000G & 1000G & 2.9 \\
Kenya & Luhya (LWK) & 101 & 4x & 1000G & 1000G & 2.2 \\
Sierra Leone & Mende (MSL) & 85 & 4x & 1000G & 1000G & 1.9 \\
Nigeria & Yoruba (YRI) & 109 & 4x & 1000G & 1000G & 2.3 \\
%Gambia & Gambian & 2 & 30x & 1000G & 1000G \\
%Nigeria & Esan & 2 & 30x & 1000G & 1000G \\
%Sierra Leone & Mende & 2 & 30x & 1000G & 1000G \\
%Kenya & Luhya & 2 & 30x & Coriell & Simons Foundation \\
%Kenya & Masai & 2 & 30x & Coriell & Simons Foundation \\
%Kenya & Luo & 2 & 30x & Geoge Ayodo & Simons Foundation \\
%Kenya & Somali & 1 & 30x & Geoge Ayodo & Simons Foundation \\
%Algeria & Mozabite & 8 & 8x & HGDP/Martin et al 2014 & NCBI-SRA \\
%DRC & Mbuti & 8 & 7x & HGDP/Martin et al 2014 & NCBI-SRA \\
%Namibia & San (Ju-hoan) & 6 & 10x & HGDP/Martin et al 2014 & NCBI-SRA \\
%Algeria & Mozabite & 2 & 30x & HGDP/Reich+Tyler-Smith & Simons Foundation and WTSI \\
%South Africa & Bantu (Tswana, Herero, Pedi, Sotho, Ovambo, Zulu) & 8 & 30x & HGDP/Reich+Tyler-Smith & Simons Foundation and WTSI \\
%South Africa & \begin{tabular}[b]{@{}l@{}}Bantu\\(Tswana, Herero, Pedi, Sotho, Ovambo, Zulu)\end{tabular} & 8 & 30x & HGDP/Reich+Tyler-Smith & Simons Foundation and WTSI \\
%Kenya & Bantu & 11 & 30x & HGDP/Reich+Tyler-Smith & Simons Foundation and WTSI \\
%CAR & Biaka Pygmy & 23 & 30x & HGDP/Reich+Tyler-Smith & Simons Foundation and WTSI \\
%Senegal & Mandenka & 22 & 30x & HGDP/Reich+Tyler-Smith & Simons Foundation and WTSI \\
%DRC & Mbuti Pygmy & 12 & 30x & HGDP/Reich+Tyler-Smith & Simons Foundation and WTSI \\
%Namibia & San (Ju-hoan) & 6 & 30x & HGDP/Reich+Tyler-Smith & Simons Foundation and WTSI \\
%Nigeria & Yoruba & 21 & 30x & HGDP/Reich+Tyler-Smith & Simons Foundation and WTSI \\
%South Africa & Khomani & 2 & 4x/9x & Kidd et al 2014 & NCBI-SRA \\
%South Africa & Khomani & 2 & 30x & Brenna Henn & Simons Foundation \\
%Sudan & Dinka & 3 & 30x & Michael Hammer & Simons Foundation \\
%Namibia & Khoisan & 1 & 10x & Schuster et al 2009 & NCBI-SRA \\
%South Africa & Xhosa/Tswana & 1 & 30x & Schuster et al 2009 & NCBI-SRA \\
%North Africa & Arab-/Berber-speaking groups & ?? & 20x & Private (David Comas) &  \\
%Morocco & Saharawi & 2 & 30x & David Comas & Simons Foundation \\
%South Africa & Mixed (Cape Coloured) & 8 & 30x & SAHGP &  \\
%South Africa & Sotho & 8 & 30x & SAHGP &  \\
%South Africa & Xhosa & 8 & 30x & SAHGP &  \\
\hline
Total* &  & 3031 &  &  & & 81.7
\end{tabular}
%\end{longtable}
%\end{landscape}
\caption{Sample sets to be included in design of the African chip array. Size refers to size of mapped sequence reads stored in bam file format.}
\label{table:samples}
\end{table}

%1000G numbers parsed from
%ftp://ftp.1000genomes.ebi.ac.uk/vol1/ftp/release/20130502/supporting/hd_genotype_chip/

%    74 ../QC/pops/LWK/LWK.postQC.autosomes.fam

%grep -f <(awk '{print substr($1,12,10)}' /lustre/scratch114/projects/uganda_gwas/QC/omni2.5-8_20120809_gwa_uganda_gtu_flipped.postQC.autosomes.fam) /lustre/scratch113/projects/agv/users/tc9/metadata/pop.dic | awk '{print $2}' | sort | uniq -c

%grep -f <(awk '{print substr($1,12,10)}' /lustre/scratch114/projects/uganda_gwas/QC/omni2.5-8_20120809_gwa_uganda_gtu_flipped.postQC.autosomes.fam) <(grep -f /lustre/scratch114/projects/ug2g/metadata/ug2g_uganda_gwas_intersection_343_samples.txt /lustre/scratch114/projects/ug2g/metadata/ug2g.2000.samples.txt) | awk '{print $4}' | sort | uniq -c | sort -k1nr,1
%    251 Baganda
%     49 Banyarwanda
%     13 RwandeseUgandan
%     12 Banyankole
%      8 Mukiga
%      7 Murundi
%      2 Mutanzania
%      1 Basoga

%grep -f <(awk '{print substr($1,12,10)}' /lustre/scratch114/projects/uganda_gwas/QC/omni2.5-8_20120809_gwa_uganda_gtu_flipped.postQC.autosomes.fam) /lustre/scratch113/projects/agv/users/tc9/QC/pops/*/*.postQC.autosomes.fam | cut -d: -f1 | sort | uniq -c | sort -k1nr,1
%    223 /lustre/scratch113/projects/agv/users/tc9/QC/pops/Banyarwanda_octo/Banyarwanda_octo.postQC.autosomes.fam
%    197 /lustre/scratch113/projects/agv/users/tc9/QC/pops/Baganda_octo/Baganda_octo.postQC.autosomes.fam
%     97 /lustre/scratch113/projects/agv/users/tc9/QC/pops/Barundi/Barundi.postQC.autosomes.fam
%     28 /lustre/scratch113/projects/agv/users/tc9/QC/pops/Baganda_quad/Baganda_quad.postQC.autosomes.fam

%grep -f <(awk '{print substr($1,12,10)}' /lustre/scratch114/projects/uganda_gwas/QC/omni2.5-8_20120809_gwa_uganda_gtu_flipped.postQC.autosomes.fam) /lustre/scratch113/projects/agv/users/tc9/metadata/hiseq2omni/uganda.intersect.dic | awk '{print $1}' | grep -w -f - /lustre/scratch113/projects/agv/users/tc9/metadata/pop.dic | awk '{print $2}' | sort | uniq -c
%     94 Baganda


\begin{table}[htp]
\centering
\resizebox{\textwidth}{!}{%
\begin{tabular}{lrlllr}
\hline
Population & Sample count & Source & Platform & Algorithm & Intersection \\
\hline
%%%\endhead % all the lines above this will be repeated on every page
%%%Uganda & 4778 & Uganda GPC & Omni2.5-8 & Illuminus & 343 UG2G and 94 AGVP \\
Baganda & 3585 & Uganda GPC & Omni2.5-8 & Illuminus & 251 UG2G and 94 AGVP \\
Banyarwanda & 422 & Uganda GPC & Omni2.5-8 & Illuminus & 49 UG2G \\
RwandeseUgandan & 202 & Uganda GPC & Omni2.5-8 & Illuminus & 13 UG2G \\
Banyankole & 147 & Uganda GPC & Omni2.5-8 & Illuminus & 12 UG2G \\
Bakiga & 60 & Uganda GPC & Omni2.5-8 & Illuminus & 8 UG2G \\
Barundi & 191 & Uganda GPC & Omni2.5-8 & Illuminus & 7 UG2G \\
Mutanzania & 44 & Uganda GPC & Omni2.5-8 & Illuminus & 2 UG2G \\
Basoga & 16 & Uganda GPC & Omni2.5-8 & Illuminus & 1 UG2G \\
Mufumbira & 5 & Uganda GPC & Omni2.5-8 & Illuminus & 0 \\
Mutooro & 10 & Uganda GPC & Omni2.5-8 & Illuminus & 0 \\
Other & 96 & Uganda GPC & Omni2.5-8 & Illuminus & 0 \\

Baganda & 90 & AGVP & Omni2.5-4 & Illuminus & 28* AGVP \\
Baganda & 197 & AGVP & Omni2.5-8 & Illuminus & 27* AGVP and 13 UG2G \\
Banyarwanda & 83 & AGVP & Omni2.5-4 & Illuminus & 0 \\
Banyarwanda & 223 & AGVP & Omni2.5-8 & Illuminus & 24 UG2G \\
Barundi & 97 & AGVP & Omni2.5-8 & Illuminus & 3 UG2G \\
Fula & 74 & AGVP & Omni2.5-8 & Illuminus & 0 \\
Ga-Adangbe & 90 & AGVP & Omni2.5-4 & Illuminus & 0 \\
Ga-Adangbe & 11 & AGVP & Omni2.5-8 & Illuminus & 0 \\
Igbo & 99 & AGVP & Omni2.5-4 & Illuminus & 0 \\
Jola & 79 & AGVP & Omni2.5-8 & Illuminus & 0 \\
Kalenjin & 100 & AGVP & Omni2.5-4 & Illuminus & 0 \\
Kikuyu & 99 & AGVP & Omni2.5-4 & Illuminus & 0 \\
Mandinka & 88 & AGVP & Omni2.5-8 & Illuminus & 0 \\
Wolof & 78 & AGVP & Omni2.5-8 & Illuminus & 0 \\
Sotho & 86 & AGVP & Omni2.5-8 & Illuminus & 0 \\
Zulu & 95 & AGVP & Omni2.5-4 & Illuminus & 95 \\
Zulu & 9 & AGVP & Omni2.5-8 & Illuminus & 0 \\
Amhara & 43 & AGVP & Omni2.5-8 & Illuminus & 24 \\
Gumuz & 0 & AGVP & Omni2.5-8 & Illuminus & 0 \\
Oromo & 26 & AGVP & Omni2.5-8 & Illuminus & 19 \\
Somali & 39 & AGVP & Omni2.5-8 & Illuminus & 20 \\
Wolayta & 0 & AGVP & Omni2.5-8 & Illuminus & 0 \\
Egypt & 114 & ADRP & Omni2.5-8 & Illuminus & 97 \\
Khoe-San (Nama) & 89 & ADRP & Omni2.5-8 & Illuminus & 89 \\

Esan (ESN) & 172 & 1000G & Affy6.0 & - & 99  \\

Gambian (GWD) & 180 & 1000G & Affy6.0 & - & 113  \\

Luhya (LWK) & 116 & 1000G & Omni2.5-8 & - & 99 \\
Luhya (LWK) & 110 & 1000G & Affy6.0 & - & 97  \\

Mende (MSL) & 122 & 1000G & Affy6.0 & - & 108 \\

Yoruba (YRI) & 189 & 1000G & Omni2.5-8 & - & 108 \\
Yoruba (YRI) & 182 & 1000G & Affy6.0 & - & 108 \\

Masai (MKK) & 31 & 1000G & Omni2.5-8 & - & 0 \\
\hline
Total* & ~6403 & & & & 
\end{tabular}
}
\caption{Sample sets available for creation of a haplotype scaffold to be used for phasing with SHAPEIT2. Sample counts refer to sample counts after quality control. Algorithm refers to the algorithm used to call the chip genotypes. Intersection refers to the count of samples at the intersection between the chip data and the sequence data. Abbreviations are Omni2.5-8 for Illumina HumanOmni2.5-8 BeadChip Kit, Affy6.0 for Affymetrix Genome-Wide Human SNP Array 6.0, AGVP for African Genome Variation Project, 1000G for 1000 Genomes Project. *The 27 and 28 Baganda samples intersecting with the AGVP sequence data are a subset of the 4778 chip genotyped Uganda GPC samples.}
\label{tab:samples_chip}
\end{table}