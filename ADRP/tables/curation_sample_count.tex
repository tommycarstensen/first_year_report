\begin{table}[htbp]
\centering
\begin{tabular}{l|r|r|r|r}
 & Sequencing & QC & verifyBamID & Heterozygosity \\
\hline
Baganda & 1667 & 1656 & 1647 & 1637 \\
Banyarwanda & 199 & 198 & 197 & 196 \\
Barundi & 51 & 51 & 51 & 51 \\
Banyankole & 36 & 36 & 36 & 36 \\
Bakiga & 30 & 30 & 30 & 30 \\
Rwandese Ugandan & 76 & 74 & 74 & 74 \\
Uganda, Other & 41 & 41 & 41 & 41 \\
Zulu & 100 & 98 & 98 & 98 \\
Wolayta & 24 & 24 & 24 & 24 \\
Somali & 24 & 24 & 24 & 24 \\
Oromo & 24 & 24 & 24 & 24 \\
Gumuz & 24 & 23 & 23 & 23 \\
Amhara & 24 & 22 & 22 & 22 \\
Egypt & 100 & 100 & 100 & 100 \\
Khoesan & 111 & 107 & 86 & 85 \\
LWK & 99 & N/A & N/A & 99 \\
GWD & 113 & N/A & N/A & 113 \\
MSL & 85 & N/A & N/A & 85 \\
ESN & 99 & N/A & N/A & 99 \\
YRI & 108 & N/A & N/A & 108 \\
1000Gp3 (nonAFR, ASW, ACB) & 2000 & N/A & N/A & 2000 \\
\hline
 &  &  &  &  \\
Sum & 5035 & 5012 & 4981 & 4969
\end{tabular}
\caption{Count of samples after each step of the curation process. N/A means that curation step was not carried out for that data set. Sequencing is the number of sequenced samples. QC is the initial QC process adapted from UK10K. verifyBamID is the subsequent check for contaminated samples with excess heterozygosity and swapped samples. Heterozygosity is the check of heterozygosity outliers after recalling variants prior to refinement with Beagle4.}
\label{tab:samplecount}
\end{table}