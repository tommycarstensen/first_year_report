\section{Results}

\subsection{Data description}
\label{sec:agv_data_description}
Related and unrelated male and female samples from 3 different African populations were genotyped using the Illumina HumanOmni2.5 genotype chip array and sequenced on an Illumina HiSeq 2500 to different depths of coverage (table \ref{tab:samples}, figure \ref{fig:coverage}). The \gls{QC} of the samples genotyped on the Illumina HumanOmni2.5-4 and 2.5-8 platforms is described elsewhere\cite{Gurdasani2015} and on page \pageref{subsec:chipQC}.
\begin{table}[h]
\centering
\resizebox{\textwidth}{!}{
\begin{tabular}{llllllll}
                          &         & \begin{tabular}[c]{@{}l@{}}HiSeq\\ samples\end{tabular} & \textless{depth}\textgreater & \begin{tabular}[c]{@{}l@{}}Omni\\ samples\end{tabular} & \begin{tabular}[c]{@{}l@{}}Sample\\ intersection\end{tabular} & \begin{tabular}[c]{@{}l@{}}HiSeq\\ variants\end{tabular} & \begin{tabular}[c]{@{}l@{}}Omni\\ variants\end{tabular} \\ \hline
Baganda                   &         & 100                                                     & 4x                            & 4778                                                   & 94                                                            & 20,461,747                                               & 2,124,005                                               \\ \hline
Zulu                      &         & 100                                                     & 4x                            & 100                                                    & 95                                                            & 20,267,592                                               & 2,050,451                                               \\ \hline
\multirow{5}{*}{Ethiopia} & Amhara  & 24                                                      & 8x                            & 46                                                     & 24                                                            & \multirow{5}{*}{20,452,231}                              & \multirow{5}{*}{2,143,095}                              \\ \cline{2-6}
                          & Gumuz   & 24                                                      & 8x                            & 0                                                      & 0                                                             &                                                          &                                                         \\ \cline{2-6}
                          & Oromo   & 24                                                      & 8x                            & 31                                                     & 19                                                            &                                                          &                                                         \\ \cline{2-6}
                          & Somali  & 24                                                      & 8x                            & 52                                                     & 20                                                            &                                                          &                                                         \\ \cline{2-6}
                          & Wolayta & 24                                                      & 8x                            & 0                                                      & 0                                                             &                                                          &                                                         \\ \hline
\end{tabular}
}
\caption[Summary statistics of sample and SNP count for SNP array and sequence data.]{Initial average sequencing depth of coverage (<depth>) for each of the 3 African populations and count of samples sequenced on the the Illumina HiSeq 2000 platform (HiSeq) and genotyped on the Illumina Omni2.5 platform (Omni2.5) after \gls{QC}. Intersection refers to the count of samples at the intersection between the two platforms. Variants refers to the number of called variants.}
\label{tab:samples}
\end{table}
Each of the two methods for obtaining genotypes yields a different number of non-monomorphic autosomal SNPs. Approximately 20\% of the SNPs from the Omni2.5 chip are monomorphic (table \ref{tab:chip_SNPs}).
\begin{table}[h]
\centering
%\resizebox{\textwidth}{!}{%
\begin{tabular}{l|ll|ll|ll}
MAF         & \multicolumn{2}{c}{Baganda} & \multicolumn{2}{c}{Ethiopia} & \multicolumn{2}{c}{Zulu} \\ \hline
0\%         & 392,140       & 17.6\%      & 384,727       & 18.8\%       & 450,793     & 21.0\%     \\ \hline
(0-1\%{]}   & 66,659        & 3.0\%       & 99,903        & 4.9\%        & 127,986     & 6.0\%      \\ \hline
(1-2\%{]}   & 113,203       & 5.1\%       & 85,894        & 4.2\%        & 86,327      & 4.0\%      \\ \hline
(2-5\%{]}   & 271,092       & 12.2\%      & 220,036       & 10.7\%       & 185,929     & 8.7\%      \\ \hline
(5-10\%{]}  & 292,272       & 13.1\%      & 270,961       & 13.2\%       & 246,133     & 11.5\%     \\ \hline
(10-20\%{]} & 400,588       & 18.0\%      & 359,861       & 17.6\%       & 343,192     & 16.0\%     \\ \hline
(20-50\%{]} & 694,304       & 31.1\%      & 629,069       & 30.7\%       & 702,735     & 32.8\%     \\ \hline
\end{tabular}
%}
\caption{Illumina Omni2.5 post QC autosomal \gls{SNP} counts in different \gls{MAF} bins.}
\label{tab:chip_SNPs}
\end{table}
The majority of the sequenced \glspl{SNP} are already present in phase 1 of \gls{1000G}\cite{1000G2012}, but novel SNPs are present in all three populations (figure \ref{fig:intersection}).

\subsection{Comparison of variant calling and refinement methods}
\subsection{Comparison of variant calling software}
\subsection{Comparison of refinement software and reference panels}

\subsection{Statistics after downsampling}

As the coverage drops the ability to call rare \glspl{SNP} decreases (figure \ref{fig:downsampling_SNP_count} and table\ref{tab:downsampling_omni_intersection}). At lower coverage the genotype correlation between the calls from the SNP array and sequencing platforms is also reduced and especially for rare variants (figure \ref{fig:downsampling_correlation_omni}.

\begin{figure}[htp]
\centering
\includegraphics[width=0.45\textwidth]{fig/downsampling_ethiopia_venn.png}
\includegraphics[width=0.45\textwidth]{fig/ethiopia_VR.png}
\caption[Down-sampling and loss of rare variants in the Ethiopian samples.]{Left: Venn diagram showing intersections between call sets at different coverages. Right: \gls{SNP} count in the Ethiopian population at different coverages in different \gls{MAF} bins after calling and filtering of variants with \gls{GATK} UnifiedGenotyper and VariantRecalibrator.\cite{DePristo2011} The y-axis is logarithmic. The ability to call rare variants is impaired at lower coverages.}
\label{fig:downsampling_SNP_count}
\end{figure}
\begin{figure}[htp]
\begin{subfigure}{.3\textwidth}
  \centering
  \includegraphics[width=1.0\linewidth]{fig/coverage_baganda.png}
  \caption{Baganda}
\end{subfigure}
\begin{subfigure}{.3\textwidth}
  \centering
  \includegraphics[width=1.0\linewidth]{fig/coverage_ethiopia.png}
  \caption{Ethiopia}
\end{subfigure}
\begin{subfigure}{.3\textwidth}
  \centering
  \includegraphics[width=1.0\linewidth]{fig/coverage_zulu.png}
  \caption{Zulu}
\end{subfigure}
\caption{Average depth of coverage in each of the 3 sequenced populations.}
\label{fig:coverage}
\end{figure}
\begin{figure}[htp]
\centering
\subfloat[a]{\includegraphics[width=0.4\linewidth]{Chapter2/fig/venn3.png}}
\subfloat[b]{\includegraphics[width=0.4\linewidth]{Chapter2/fig/venn3_complement.png}}
\subfloat[c]{\includegraphics[width=0.4\linewidth]{Chapter2/fig/venn3_stackplot_SNPs.png}}
\subfloat[d]{\includegraphics[width=0.4\linewidth]{Chapter2/fig/venn3_stackplot_complement_1000G_SNPs.png}} 
\caption{Sharing of variants between populations. a) SNP intersection between the 3 sequenced populations subsampled to 100 samples and downsampled to 4x coverage. b) Intersection between the novel variants in the 3 populations; i.e. variants in the relative complement of \gls{1000G} phase 1 with respect to the 3 populations. c) Relative allele frequency spectra for variants in different sets of the Venn diagrams depicted in a and b. The majority of the novel variants with respect to \gls{1000G} are unique to each population. Figures were created with eulerAPE\cite{10.1371/journal.pone.0101717} and matplotlib\cite{Hunter2007}}
\label{fig:intersection}
\end{figure}
\begin{table}[htp]
\centering
\begin{tabular}{r|rrrr}
{\gls{MAF}} & {4x} & {2x} & {1x} & {0.5x} \\ \hline
0\%                        & 6.7                       & 5.3                       & 4.4                       & 4.0                         \\
(0-1\%{]}                  & 55.1                      & 28.7                      & 14.1                      & 8.0                         \\
(1-2\%{]}                  & 78.9                      & 52.4                      & 29.2                      & 15.0                        \\
(2-5\%{]}                  & 91.0                      & 80.9                      & 61.6                      & 37.2                        \\
(5-10\%{]}                 & 94.2                      & 91.4                      & 87.1                      & 70.5                        \\
(10-20\%{]}                & 95.2                      & 93.6                      & 94.1                      & 91.4                        \\
(20-50\%{]}                & 95.3                      & 95.1                      & 95.6                      & 97.1 \\ \hline
\end{tabular}
\caption{Intersection between variants for the 100 Baganda samples on the Omni2.5 platform and the HiSeq2000 platform at different coverages as a percentage of variants on the Omni2.5 platform. Fewer rare SNP array genotyped variants are called at lower coverage.}
\label{tab:downsampling_omni_intersection}
\end{table}

\begin{figure}[htp]
\centering
\includegraphics[width=0.45\textwidth]{fig/downsampling_ethiopia_venn.png}
\includegraphics[width=0.45\textwidth]{fig/ethiopia_VR.png}
\caption[Down-sampling and loss of rare variants in the Ethiopian samples.]{Left: Venn diagram showing intersections between call sets at different coverages. Right: \gls{SNP} count in the Ethiopian population at different coverages in different \gls{MAF} bins after calling and filtering of variants with \gls{GATK} UnifiedGenotyper and VariantRecalibrator.\cite{DePristo2011} The y-axis is logarithmic. The ability to call rare variants is impaired at lower coverages.}
\label{fig:downsampling_SNP_count}
\end{figure}

\subsection{Apparent sample size}

Figure - Apparent sample sizes for common, rare and very rare variants without data processing costs. 
Table - Costs associated with acquiring genotypes via two different platforms.