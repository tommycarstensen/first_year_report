\section{Introduction}
There is a greater genetic diversity in humans in sub Saharan Africa than anywhere else on the planet.\cite{Bowcock1994}\cite{Jorde1995}\cite{Tishkoff08031996}\cite{Jorde2000}\cite{Stephens2001}\cite{Tishkoff2002}\cite{Tishkoff2004}\cite{HapMap2005}\cite{Ramachandran01112005}\cite{Tishkoff22052009}\cite{1000G2010}\cite{1000G2012} Due to the recent African origin of modern humans the African continent hosts the largest number of unique variants and thus the best possibility of discovering rare variants affecting common diseases and complex traits. Despite these facts most populations from which 100 samples or more have been characterized by \gls{WGS} are Caucasian/European and the majority of current commercial genotype chip arrays are designed based on Caucasian/European populations. And this is also despite the importance of using an appropriate reference panel for an African Malaria GWAS having been previously shown.\cite{2009Jallow} To inform the design of genetic studies in Africa, we genotyped and sequenced samples from three distinct sub Saharan African populations in the East (Baganda), South (Zulu) and North (Ethiopia) and generated a continent-specific reference panel.