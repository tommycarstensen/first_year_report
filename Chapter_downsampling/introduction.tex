\section{Introduction}
\label{sec:downsampling_introduction}

Two main platforms exist for acquiring genotypes from across the genome; one is sequencing, which yields tens of millions of known and novel variants depending on population diversity and samples size and another is SNP array genotyping, which provides accurate genotypes for a subset of sites known to be variant in some populations. For purposes of comparing the cost effectiveness of study designs in Africa we obtained genotypes from both platforms. We genotyped and sequenced samples from three distinct sub Saharan African populations in the East (Baganda), South (Zulu) and North (Ethiopia).

Extremely low coverage sequencing, low coverage sequencing and SNP array approaches have already been evaluated in terms of ability to capture common variation in European populations for \glspl{GWAS},\cite{Pasaniuc2012}\cite{10.1371/journal.pcbi.1002604} but this has not been adressed in African populations. I do so here.

Sequencing and \gls{SNP} chip genotyping are both able to provide genotype information on individuals from a population. They do so at different costs, different levels of quality, different density and with different downstream requirements on computational and human resources. In this chapter I investigate the cost effectiveness of different study designs by comparison of sequencing data and chip data obtained from the same set of individuals. The ability to sequence depends on the sequence content, and the ability to map sequence reads depends on the uniqueness of the sequence and the presence and length of indels. In addition to the \gls{SNP} array density it is also important whether a population specific \gls{SNP} array and a population specific reference panel is used when imputing to achieve a greater SNP density, which allows for fine mapping. In chapter \ref{ch:reference_panel} I show that a continent specific array outperforms a 1000G reference panel in terms of imputation accuracy. In chapter \ref{ch:downsampling} I only focus on identifying the cheapest study design. I do so by comparison of sequence data generated on the Illumina HiSeq2000 platform and SNP array data generated on the Illumina HumanOmni2.5-4 (quad) and -8 (octo) platforms. Specifically I look at the ability to call rare and common variants at lower coverage (figure \ref{fig:downsampling_SNP_count} on page \pageref{fig:downsampling_SNP_count}) and the correlation with SNP array data at different coverages (table \ref{tab:downsampling_omni_intersection} on page \pageref{tab:downsampling_omni_intersection}).

I have carried out all the analytical work described in this section not related to sample collection and laboratory work with the exception of the QC and pre-processing of the raw sequence reads and the calculation of F$_{st}$ values. This work was carried out by Martin Pollard and Savita Karthikeyan, respectively.

\subsection{Data description}
\label{sec:downsampling_datadescription}
Samples from 3 different African populations were genotyped using the Illumina HumanOmni2.5 SNP array and sequenced on an Illumina HiSeq 2500 sequencing machine (tables \ref{tab:short_summary} and \ref{tab:samples} on page \pageref{tab:samples}).

\begin{table}[ht]
\centering
\begin{tabular}{l|l|l|l|l}
Region & Country & Population & Language & Language family \\ \hline
East Africa & Uganda\cite{Asiki01022013} & Baganda & Luganda & Great Lakes Bantu \\ \hline
Southern Africa & South Africa & Zulu & Zulu & Nguni, Southern Bantu \\ \hline
\multirow{5}{*}{Northeast Africa} & \multirow{5}{*}{Ethiopia} & Amhara & Amharic & Semitic, Afro-Asiatic \\
& & Gumuz & Gumuz & Nilo-Saharan? \\
& & Oromo & Oromo & Cushitic, Afro-Asiatic \\
& & Somali & Somali & Cushitic, Afro-Asiatic \\
& & Wolayta & Wolayta & Omotic, Afro-Asiatic \\
\hline
\end{tabular}
\caption{Short summary of the 3 cohorts from which sequencing and SNP array data was obtained.}
\label{tab:short_summary}
\end{table}