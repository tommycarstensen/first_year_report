\section{Results}


\subsection{Data description after QC}
\label{sec:agv_data_description}
Related and unrelated male and female samples from 3 different African populations (table \ref{tab:short_summary}) were genotyped using the Illumina HumanOmni2.5 genotype chip array and sequenced on an Illumina HiSeq 2000 to different depths of coverage (table \ref{tab:samples}) and subsequently downsampled to a similar coverage of 4x (figure \ref{fig:coverage}). The \gls{QC} of the samples genotyped on the Illumina HumanOmni2.5-4 and 2.5-8 platforms is described elsewhere\cite{Gurdasani2015} and on page \pageref{subsec:chipQC}.
\begin{table}[h]
\centering
\resizebox{\textwidth}{!}{
\begin{tabular}{llllllll}
                          &         & \begin{tabular}[c]{@{}l@{}}HiSeq\\ samples\end{tabular} & \textless{depth}\textgreater & \begin{tabular}[c]{@{}l@{}}Omni\\ samples\end{tabular} & \begin{tabular}[c]{@{}l@{}}Sample\\ intersection\end{tabular} & \begin{tabular}[c]{@{}l@{}}HiSeq\\ variants\end{tabular} & \begin{tabular}[c]{@{}l@{}}Omni\\ variants\end{tabular} \\ \hline
Baganda                   &         & 100                                                     & 4x                            & 4778                                                   & 94                                                            & 20,461,747                                               & 2,124,005                                               \\ \hline
Zulu                      &         & 100                                                     & 4x                            & 100                                                    & 95                                                            & 20,267,592                                               & 2,050,451                                               \\ \hline
\multirow{5}{*}{Ethiopia} & Amhara  & 24                                                      & 8x                            & 46                                                     & 24                                                            & \multirow{5}{*}{20,452,231}                              & \multirow{5}{*}{2,143,095}                              \\ \cline{2-6}
                          & Gumuz   & 24                                                      & 8x                            & 0                                                      & 0                                                             &                                                          &                                                         \\ \cline{2-6}
                          & Oromo   & 24                                                      & 8x                            & 31                                                     & 19                                                            &                                                          &                                                         \\ \cline{2-6}
                          & Somali  & 24                                                      & 8x                            & 52                                                     & 20                                                            &                                                          &                                                         \\ \cline{2-6}
                          & Wolayta & 24                                                      & 8x                            & 0                                                      & 0                                                             &                                                          &                                                         \\ \hline
\end{tabular}
}
\caption[Summary statistics of sample and SNP count for SNP array and sequence data.]{Initial average sequencing depth of coverage (<depth>) for each of the 3 African populations and count of samples sequenced on the the Illumina HiSeq 2000 platform (HiSeq) and genotyped on the Illumina Omni2.5 platform (Omni2.5) after \gls{QC}. Intersection refers to the count of samples at the intersection between the two platforms. Variants refers to the number of called variants.}
\label{tab:samples}
\end{table}
\begin{figure}[htp]
\begin{subfigure}{.3\textwidth}
  \centering
  \includegraphics[width=1.0\linewidth]{fig/coverage_baganda.png}
  \caption{Baganda}
\end{subfigure}
\begin{subfigure}{.3\textwidth}
  \centering
  \includegraphics[width=1.0\linewidth]{fig/coverage_ethiopia.png}
  \caption{Ethiopia}
\end{subfigure}
\begin{subfigure}{.3\textwidth}
  \centering
  \includegraphics[width=1.0\linewidth]{fig/coverage_zulu.png}
  \caption{Zulu}
\end{subfigure}
\caption{Average depth of coverage in each of the 3 sequenced populations.}
\label{fig:coverage}
\end{figure}
Each of the two methods for obtaining genotypes yields a different number of non-monomorphic autosomal SNPs. Approximately 20\% of the SNPs from the Omni2.5 chip are monomorphic (table \ref{tab:chip_SNPs}).
\begin{table}[h]
\centering
%\resizebox{\textwidth}{!}{%
\begin{tabular}{l|ll|ll|ll}
MAF         & \multicolumn{2}{c}{Baganda} & \multicolumn{2}{c}{Ethiopia} & \multicolumn{2}{c}{Zulu} \\ \hline
0\%         & 392,140       & 17.6\%      & 384,727       & 18.8\%       & 450,793     & 21.0\%     \\ \hline
(0-1\%{]}   & 66,659        & 3.0\%       & 99,903        & 4.9\%        & 127,986     & 6.0\%      \\ \hline
(1-2\%{]}   & 113,203       & 5.1\%       & 85,894        & 4.2\%        & 86,327      & 4.0\%      \\ \hline
(2-5\%{]}   & 271,092       & 12.2\%      & 220,036       & 10.7\%       & 185,929     & 8.7\%      \\ \hline
(5-10\%{]}  & 292,272       & 13.1\%      & 270,961       & 13.2\%       & 246,133     & 11.5\%     \\ \hline
(10-20\%{]} & 400,588       & 18.0\%      & 359,861       & 17.6\%       & 343,192     & 16.0\%     \\ \hline
(20-50\%{]} & 694,304       & 31.1\%      & 629,069       & 30.7\%       & 702,735     & 32.8\%     \\ \hline
\end{tabular}
%}
\caption{Illumina Omni2.5 post QC autosomal \gls{SNP} counts in different \gls{MAF} bins.}
\label{tab:chip_SNPs}
\end{table}
The majority of the sequenced \glspl{SNP} are already present in phase 1 of \gls{1000G}\cite{1000G2012}, but novel SNPs are present in all three populations (figure \ref{fig:intersection}).

\begin{figure}[htp]
\centering
\subfloat[a]{\includegraphics[width=0.4\linewidth]{Chapter2/fig/venn3.png}}
\subfloat[b]{\includegraphics[width=0.4\linewidth]{Chapter2/fig/venn3_complement.png}}
\subfloat[c]{\includegraphics[width=0.4\linewidth]{Chapter2/fig/venn3_stackplot_SNPs.png}}
\subfloat[d]{\includegraphics[width=0.4\linewidth]{Chapter2/fig/venn3_stackplot_complement_1000G_SNPs.png}} 
\caption{Sharing of variants between populations. a) SNP intersection between the 3 sequenced populations subsampled to 100 samples and downsampled to 4x coverage. b) Intersection between the novel variants in the 3 populations; i.e. variants in the relative complement of \gls{1000G} phase 1 with respect to the 3 populations. c) Relative allele frequency spectra for variants in different sets of the Venn diagrams depicted in a and b. The majority of the novel variants with respect to \gls{1000G} are unique to each population. Figures were created with eulerAPE\cite{10.1371/journal.pone.0101717} and matplotlib\cite{Hunter2007}}
\label{fig:intersection}
\end{figure}

%\subsubsection{Comparison of variant calling and refinement methods}
%\subsubsection{Comparison of variant calling software}
%\subsubsection{Comparison of refinement software and reference panels}


\subsection{Comparison of whole genome sequencing and SNP array designs}
\label{subsec:result_downsampling_sensspec}
The sensitivity and specificity was evaluated for each of the study designs as described on page \pageref{subsec:sensspec}. The sensitivity of all study designs was greater than 0.95 for common variants (\gls{MAF} greater than 0.05) with respect to 8x for Ethiopians (figure \ref{fig:SN12f3}) and 4x sequence data for all populations (\ref{fig:SN12f4}). For variants with a \gls{MAF} less than 0.05 the sensitivity was lower. The sensitivity for singletons and novel variants relative to 1000G phase 1 was even lower, which is expected, since very few reads will support singletons. The sensitivity of the imputed SNP array data (Omni2.5) is comparable to that of the 0.5x coverage design. However, the disadvantage of the \gls{SNP} array design is the inability to acquire genotypes of novel variants, but this will ineviatiably become less relevant as the size and diversity of reference panels for imputation grow. All designs had high specificity relative to the higher coverage reference sequence data (8x figure \ref{fig:SN12f3} and 4x figure \ref{fig:SN12f4}), which indicates a low \gls{FPR} or identical false positives in the reference data.

\begin{figure}
\centering
\includegraphics[trim={0 0cm 0 0cm},clip,width=0.8\textwidth]{fig/SN12f3}
\caption[Sensitivity and specificity of different designs relative to 8x coverage data.]{Sensitivity and specificity of different designs (2x, 1x, 0.5x \gls{ULC} sequencing, Omni 2.5M genotyping with imputation) with respect to 4x coverage among Ethiopian individuals.}
\label{fig:SN12f3}
\end{figure}
\begin{figure}
\centering
\includegraphics[trim={0 4cm 0 0cm},clip,width=0.8\textwidth]{fig/SN12f4}
\caption[Sensitivity and specificity for each of 3 populations relative to 4x after down-sampling.]{Same as figure \ref{fig:SN12f4} relative to 4x data; i.e. sensitivity and specificity for different \gls{ULC} sequencing designs, and imputed data from the Omni 2.5M \gls{SNP} array relative to 4x \gls{WGS} data from populations Zulu (top), Uganda (middle) and Ethiopia (bottom). Sensitivity and specificity is defined in the text and illustrated in figure \ref{fig:SN12f2} on page \pageref{fig:SN12f2}.}
\label{fig:SN12f4}
\end{figure}


\subsection{Data description after downsampling}
\label{subsec:result_downsampling}

As the coverage drops the ability to call rare \glspl{SNP} decreases (figure \ref{fig:downsampling_SNP_count} and table\ref{tab:downsampling_omni_intersection}). At lower coverage the genotype correlation between the calls from the SNP array and sequencing platforms is also reduced and especially for rare variants (figure \ref{fig:SN12f5}). Despite the correlations not being perfect for lower coverage designs for rare and common variants (figure \ref{fig:SN12f5}) the apparent sample size is still greater due to the greater sample size and in the case of common variants in particular compensates for inaccuracies in identifying true genotypes (figure \ref{fig:SN12f6} on page \pageref{subsec:result_apparent}).

\begin{figure}[htp]
\centering
\includegraphics[width=0.45\textwidth]{fig/downsampling_ethiopia_venn.png}
\includegraphics[width=0.45\textwidth]{fig/ethiopia_VR.png}
\caption[Down-sampling and loss of rare variants in the Ethiopian samples.]{Left: Venn diagram showing intersections between call sets at different coverages. Right: \gls{SNP} count in the Ethiopian population at different coverages in different \gls{MAF} bins after calling and filtering of variants with \gls{GATK} UnifiedGenotyper and VariantRecalibrator.\cite{DePristo2011} The y-axis is logarithmic. The ability to call rare variants is impaired at lower coverages.}
\label{fig:downsampling_SNP_count}
\end{figure}
\begin{table}[htp]
\centering
\begin{tabular}{r|rrrr}
{\gls{MAF}} & {4x} & {2x} & {1x} & {0.5x} \\ \hline
0\%                        & 6.7                       & 5.3                       & 4.4                       & 4.0                         \\
(0-1\%{]}                  & 55.1                      & 28.7                      & 14.1                      & 8.0                         \\
(1-2\%{]}                  & 78.9                      & 52.4                      & 29.2                      & 15.0                        \\
(2-5\%{]}                  & 91.0                      & 80.9                      & 61.6                      & 37.2                        \\
(5-10\%{]}                 & 94.2                      & 91.4                      & 87.1                      & 70.5                        \\
(10-20\%{]}                & 95.2                      & 93.6                      & 94.1                      & 91.4                        \\
(20-50\%{]}                & 95.3                      & 95.1                      & 95.6                      & 97.1 \\ \hline
\end{tabular}
\caption{Intersection between variants for the 100 Baganda samples on the Omni2.5 platform and the HiSeq2000 platform at different coverages as a percentage of variants on the Omni2.5 platform. Fewer rare SNP array genotyped variants are called at lower coverage.}
\label{tab:downsampling_omni_intersection}
\end{table}
\begin{figure}
        \centering
        \begin{subfigure}[b]{0.75\textwidth}
                \includegraphics[width=\textwidth]{fig/SN12f5a}
                \caption{Uganda}
                \label{fig:SN12f5uganda}
        \end{subfigure}%

        \begin{subfigure}[b]{0.75\textwidth}
                \includegraphics[width=\textwidth]{fig/SN12f5b}
                \caption{Zulu}
                \label{fig:zulu}
        \end{subfigure}

        \begin{subfigure}[b]{0.75\textwidth}
                \includegraphics[width=\textwidth]{fig/SN12f5c}
                \caption{Ethiopia}
                \label{fig:ethiopia}
        \end{subfigure}
        \caption[Correlation with imputed chip data and sequence data after downsampling to lower coverage.]{Correlations between different designs and overlapping genotype data from the Omni 2.5M array for each of the 3 populations; Uganda (top), Zulu (middle) and Ethiopia (bottom). The line labelled "Omni 2.5M imputed" for each design represents the correlation between data imputed from the Omni 2.5M array using the \gls{1000G} phase 1 v3 integrated panel, and the original data genotyped on the chip array.}
        \label{fig:SN12f5}
\end{figure}


\subsection{Apparent sample size}
\label{subsec:result_apparent}

By using the costs in table \ref{tab:costs} and equation \ref{eq:sample_size} and the calculated correlations we calculated the apparent sample size for different study designs at a fixed cost. When not taking into consideration computational costs the ultra low coverage study design is a good choice for capturing common variation at a low cost.

\begin{figure}[htbp]

        \centering

        \begin{subfigure}[b]{0.45\textwidth}
                \includegraphics[trim={14.5cm 17.5cm 1.5cm 4.75cm},clip,width=0.95\textwidth]{fig/SN12f6}
                \caption{Uganda}
                \label{fig:SN12f6uganda}
        \end{subfigure}%

        \begin{subfigure}[b]{0.45\textwidth}
                \includegraphics[trim={0 17cm 16cm 4.75cm},clip,width=0.9\textwidth]{fig/SN12f6}
                \caption{Zulu}
                \label{fig:SN12f6zulu}
        \end{subfigure}

        \begin{subfigure}[b]{0.45\textwidth}
                \includegraphics[trim={0 2cm 16cm 18cm},clip,width=0.9\textwidth]{fig/SN12f6}
                \caption{Ethiopia}
                \label{fig:SN12f6ethiopia}
        \end{subfigure}

        \caption[
        Effective samples sizes for down-sampled and imputed chip data for common, rare and very rare variants.]{
        Effective sample sizes for studies examining associations with common (MAF>5\%) and rare (MAF<5\%) variants using different designs. Effective sample size was calculated for a fixed budget of £67,100, equivalent to current sequencing and data curation costs for \gls{WGS} of 100 samples at 4x coverage. Costs assumed here include costs of library preparation, sequencing, and computational costs (£671/sample at 4x coverage, £361 per sample at 2x, £190 per sample at 1x and £105/sample at 0.5x, and £5,368 for a 48 sample Omni 2.5-8 kit. Effective sample sizes are calculated as \textit{N}\textit{r}\textsuperscript{2}, where \textit{N} is the number of samples that can be sequenced at a fixed budget and \textit{r}\textsuperscript{2} is the accuracy of capture of genotypes, calculated empirically by the correlation between sequencing data and Omni 2.5M chip array data. For the Omni 2.5M chip array,we examine the correlation between imputed data, and genotypes on the chip array. We find a steady increase in effective sample size for a fixed budget, with lower sequencing coverage, with Omni 2.5M designs performance broadly equivalent to 0.5x sequencing.
        }

        \label{fig:SN12f6}

\end{figure} % apparent sample size
