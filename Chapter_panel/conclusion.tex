\section{Conclusions}

The reduction in imputation accuracy upon reduction of \gls{SNP} density (figure \ref{fig:SN10f2} on page \pageref{fig:SN10f2}) argues for the development of denser arrays that better capture the \gls{LD} structure in Africa (page \pageref{ch:chip_design}).

The greater reduction in imputation accuracy upon reduction of \gls{SNP} density for populations poorly represented by the reference panel used for imputation (figure \ref{fig:SN10f3} on page \pageref{fig:SN10f3}) highlights the need for a reference panel better representing the African continent. The same conclusion is reached from observing the greater improvement in imputation accuracy for the same set of populations (figure \ref{fig/SN11f1} on page \pageref{fig/SN11f1}). However, it is also noted that several Bantu populations are already well represented by an already existing Bantu population (figure \ref{fig/SN11f1}), which argues that future strategies for obtaining haplotype diversity is to sequence population isolates such as for example hunter gatherer groups.