\section{Methods}
\label{sec:methods_panel}


\subsection{Phasing of refined genotypes}
We refined the called genotypes as previously described on page \pageref{subsec:AGVrefinement}.
We phased the Beagle3 refined genotypes from the 3 populations with SHAPEIT2\cite{Delaneau2012} (release 644) for the 22 autosomes. This was preceded by a file format conversion from the Beagle3 format\footnote{http://faculty.washington.edu/browning/beagle/beagle\textunderscore3.3.2\_31Oct11.pdf} to the Oxford .gen/.sample format\footnote{https://mathgen.stats.ox.ac.uk/genetics\_software/shapeit/shapeit.html\#gensample}. We ran SHAPEIT2 across entire chromosomes and we ran with the HapMap II\cite{hapmap2007} estimated African population size\cite{Wright01031931}\cite{Wright1938}; i.e. -{}-effective-size 17469. Otherwise we used default settings and ran the default 7 burn-in, 8 pruning and 20 main iterations. We used the default 100 states per window and used default window sizes of 2\gls{Mbp}. We used the HapMap II genetic maps for estimation of recombination rates.


\subsection{Merger of haplotype reference panels}
\label{subsec:panel_merger}
After phasing and conversion of the 640 SHAPEIT2 generated haplotypes from the .haps/.sample format\footnote{https://mathgen.stats.ox.ac.uk/genetics\_software/shapeit/shapeit.html\#hapsample} to the IMPUTE2 .hap/.legend/.sample format\footnote{https://mathgen.stats.ox.ac.uk/genetics\_software/shapeit/shapeit.html\#haplegsample} we merged the reference panel with the existing \gls{1000G} phase 1 reference panel using the -{}\-merge\_ref\_panels\_output\_ref\footnote{https://mathgen.stats.ox.ac.uk/impute/impute\_v2.html\#\-merge\_ref\_panels\_output\_ref} option.
The method is well documented online.\footnote{https://mathgen.stats.ox.ac.uk/impute/impute\_v2.html\#merging\_panels}
We ran on 2\gls{Mbp} fragments. We ran with the following default settings:
\begin{itemize}
\item \-khap 500\footnote{https://mathgen.stats.ox.ac.uk/impute/impute\_v2.html\#\-k\_hap} (number of reference haplotypes to use as templates when imputing missing genotypes, which is the only parameter we expect to significantly change results if increased from the default.)
\item \-Ne 17649\footnote{https://mathgen.stats.ox.ac.uk/impute/impute\_v2.html\#\-Ne} (effective population size\cite{Wright01031931}\cite{Wright1938})
\item \-buffer 250 \gls{kbp} (buffer region to include on each side of the analysis interval to prevent imputation quality from deteriotating near the edges)
\item \-iter 30 (total number of iterations to perform)
\item \-burnin 10 (number of iterations to discard, which don't contribute to the final imputation probabilities)
\item \-k 80 (number of haplotypes to use as templates when phasing observed genotypes)
\end{itemize}

\begin{table}[htbp]
\centering
\begin{tabular}{l|r|r}
 & samples & variants \\ \hline
 3 populations & 320 & 34,536,882 \\
 1000G phase 1 & 1,092 & 40,504,208 \\
 Merged & 1,412 & 37,267,523 \ %40163067 without indel+mono removal? Too few!!!? Less than 1000G
\end{tabular}
\caption[Sample and SNP count of reference panels before and after merger.]{Contents of autosomal reference panels before and after merger. Variant count includes SNPs and indels. Indels and monomorphic variants were removed from the merged reference panel.}
\label{tab:SNP_count_merger}
\end{table}


\subsection{SNP Chip Array QC and post-processing}

For evaluation of the imputation accuracy using an expanded African reference panel we imputed SNP array data from the \gls{AGV} populations representing all the major African language groups (SM table 1 on page 3 of the supplementary material of the \gls{AGV} paper).\cite{Gurdasani2015}
The \gls{SNP} \gls{QC} was carried out as previously described (page \pageref{subsec:chipQC}). The sample and \gls{SNP} count after each \gls{QC} step can be found in SM tables 3 and 4 on pages 5 and 6 of the \gls{AGV} paper.\cite{Gurdasani2015}


\subsection{Assesment of imputation accuracy}
Imputation was carried out with the software IMPUTE2. The r2 metric reported by IMPUTE2 was used as the metric of imputation accuracy. Because the sample count was different for each population we calculated mean \gls{r2} within \gls{MAF} bins of 0.05 (figures \ref{fig:SN10f1} and \ref{fig:SN10f2}).