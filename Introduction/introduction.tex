%\chapter*{Introduction}
\chapter{Exploratory analysis of genetic studies in Africa}

\section{Introduction}

We are interested in studying genetics in Africa, because the continent has a greater genetic diversity than any other because of the founder effect and the origin of the human species in Africa\cite{1000G2012}\cite{Gurdasani2015}\cite{Tishkoff22052009}\cite{Stringer2003}. Because \glspl{NCD} are on the rise as the most common cause of death in \gls{SSA} and because \gls{HIV} and \gls{AIDS} is expected to still be the most common cause of death in \gls{SSA} in 2030 according to the \gls{WHO}\cite{10.1371/journal.pmed.0030442} % http://www.who.int/healthinfo/global_burden_disease/projections/en/
my PhD will focus mostly on analysis of DNA from individuals from \gls{SSA}, which are under this disease burden. Specifically I will work with samples from a general population cohort study in Uganda\cite{Asiki01022013} of which some have been genotyped on a dense SNP array and others sequenced to low coverage.

Because genotyping platforms designed for \glspl{GWAS} of European populations are not of equal utility in African populations due to sample selection bias in the design of these \gls{SNP} arrays, I will seek to design a \gls{SNP} array specific to the African continent as part of my PhD, which takes into account the different \gls{LD} structure in African populations. This work will be carried out in collaboration with \acrfull{H3A} and will enables researchers in Africa to carry out cost effective \glspl{GWAS}. I present an analysis plan in chapter \ref{chap:chip_design} regarding the design of such a \gls{SNP} array.

Genetics is moving from \gls{GWAS} to functional characterisation of \glspl{SNP}. Therefore I wish to carry out a statistical evaluation the structural effects of \gls{ns} \glspl{SNP} at the \gls{PPI} level. I briefly describe these plans in chapter \ref{chap:future}. I also intend to study the introgression of viral DNA into human DNA. I briefly describe this in chapter \ref{chap:future}.

There is a greater genetic diversity in humans in sub Saharan Africa than anywhere else on the planet.\cite{Bowcock1994}\cite{Jorde1995}\cite{Tishkoff08031996}\cite{Jorde2000}\cite{Stephens2001}\cite{Tishkoff2002}\cite{Tishkoff2004}\cite{HapMap2005}\cite{Ramachandran01112005}\cite{Tishkoff22052009}\cite{1000G2010}\cite{1000G2012} Due to the recent African origin of modern humans the African continent hosts the largest number of unique variants and thus the best possibility of discovering rare variants affecting common diseases and complex traits. Despite these facts most populations from which 100 samples or more have been characterized by \gls{WGS} are Caucasian/European, the majority of current commercial genotype chip arrays are designed based on Caucasian/European populations, few \glspl{GWAS} have been conducted in African populations and African populations constitute a relatively small proportion of miscellaneous global sequencing efforts such as 1000G.\cite{1000G2012} To inform the design of genetic studies in Africa, we genotyped and sequenced samples from three distinct sub Saharan African populations in the East (Baganda), South (Zulu) and North (Ethiopia) and generated a continent-specific reference panel. One of the reasons for designing a haplotype reference panel is that the accuracy of imputation and the accuracy of downstream analyses - e.g. the ability to identify novel association signals in a \gls{GWAS} - improve significantly, which has been shown previously.\cite{2009Jallow}\cite{Gurdasani2015} The presence of African populations in existing reference panels is sparse. To generate an improved reference panel we used low coverage data (4x) from the 3 populations (page \pageref{sec:agv_data_description}) to generate a reference panel merged with phase 1 of 1000G.

Sequencing and \gls{SNP} chip genotyping are both able to provide genotype information on individuals from a population. They do so at different costs, at different levels of quality, at different density and with different downstream requirements on computational and human resources. In this chapter I investigate these aspects. The ability to sequence also depends on the sequence content and the ability to map sequence reads depends on the uniqueness of the sequence and the presence and length of indels. In addition to the \gls{SNP} array density it is also important, whether a population specific \gls{SNP} array and a population specific reference panel is used when imputing to achieve a greater SNP density, which allows for fine mapping. In chapter \ref{chap:reference_panel} I show that a continent specific array outperforms a 1000G reference panel in terms of imputation accuracy. In this chapter I only focus on identifying the cheapest study design. I do so by comparison of sequence data generated on the Illumina HiSeq2000 platform and SNP array data generated on the Illumina HumanOmni2.5-4 (quad) and -8 (octo) platforms. Specifically I look at the ability to call rare and common variants at lower coverage (figure \ref{fig:downsampling_SNP_count}), the correlation with SNP array data at different coverages (figure \ref{tab:downsampling_omni_intersection}).